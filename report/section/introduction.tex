\section{Introduction}

\subsection{Choice of Project}
\emph{An original network economics research}

\subsection{Project Scenarios}

For the year 2020, people are still suffering from the COVID-19 pandemic. It brings severe challenge to the operation of traditional service industry. The aviation industry is one of those most impacted under the pandemic, since protective measures against infection could set tight restriction to the capacity of the transportation network. For example, governments around the world have prohibited cross-country transportation and the market share of the airline business has shrunk since then. Some methodologies like an event-driven approach have been implemented to quantitatively analyze the economic influence, especially after three major COVID-19 announcements were made by World Health Organization (WHO) \cite{maneenop2020impacts}. Other studies focus on scenarios variation after post-COVID-19 and how airlines react and recover from the pandemic in the aspect of world airline network (WAN) \cite{ye2020scenarios}. Aviation network features like temporal characteristics can also influence the infection rate \cite{scire2017}. Among the countries that waged arduous struggles to the pandemic, China is one of those achieved early recovery. The "Five One" policy adopted by the Civil Aviation Administration of China (CAAC)\footnote{Official website: \url{www.caac.gov.cn/en}} to prevent imported cases could play a big role. It allows mainland carriers to fly just one flight a week on one route to any other country and foreign airlines to operate just one flight a week to China. From the perspective of airline markets, some foreign airlines also make strategic response to the pandemic crisis and outline key implications for post-COVID-19 competitive landscape, to raise attention and provide recommendations for policy makers \cite{albers2020european} \cite{budd2020european}. Inspired by such fusing measures and response behavior, we further expand the decision making process into an air traffic control problem. Here we try to provide similar policies for cities in the transportation network, considering both the infection rate and passengers' utility.


\subsection{Goals}
To achieve our goals, we try to answer the following questions:
\begin{itemize}    
    \item How should we model the aviation network under the pandemic?
    \item What's the best strategy for each city/for the global network?
    \item Is there any penalty of anarchy and how do we describe it?
\end{itemize}

% \begin{itemize}
%     \item 
%     How should we model the aviation network under the pandemic? 
%     \item 
%     What's the best strategy for each city/for the global network?
%     \item
%     Is there any penalty of anarchy and how do we describe it?
% \end{itemize}

These are novel questions to solve. The solution to the first question is the foundation for our further discussion, and we try to carry out exploratory experiments seeking the solution to the second question. The third question is hard to solve and requires further distribution in such area. The main contributions of this project are the following:
\begin{itemize}
    \item We propose a new decision making model on airline capacity considering both the passengers' demand and the infection severity of different cities.
    \item We mathematically simplify our model and make it solvable under linear constraint in \emph{IBM CPLEX} solver. We also apply the Differential Evolution Algorithm using the \emph{GeatPy}\cite{geatpy} toolkit to solve our problem and find better result.
    
    \item Empirical experiments have been done to show the necessity to ban the airlines from cities with high infection severity and the difficulty to strike a balance between the demand of passengers and public health. 
\end{itemize}

The paper is organized as follows. Section 2 reviews the related work. Section 3 introduces our models, while Section 4 presents our algorithms. Section 5 reports the empirical experiments and results, followed by the discussion and conclusions in Section 6. Lastly, Section 7 briefly introduces the contributions by each group member. 

