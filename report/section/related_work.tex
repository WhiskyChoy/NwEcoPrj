% =================================

\section{Related Work}
Recently, researchers have done broad and in-depth studies focusing on how COVID-19 affected our daily life from the perspective of economic, culture and social behaviors. Treating the process of COVID-19 as a susceptible-infected model, Jia Wangping\cite{wangping2020extended} has presented a study in which, COVID-19 data from Jan 22, 2020, to Mar 16, 2020, has been used in time series form for spreading analysis. In the aspects of public transportation, a variety of effctive measures have been successfully implemented to control the COVID-19 \cite{shen2020prevention,kucharski2020early}. In our paper, decision making of dynamic airline rearrangement is actually a network optimization problem. Shangyao Yan \cite{yan2002passenger} develops an integer multiple commodity network model and a solution algorithm to help carriers simultaneously solve for better fleet routes and appropriate timetables. Some heuristic algorithms such as genetic algorithm \cite{kolker2015using} and dynamic programming \cite{khoo2014bi} have also beem applied to solve airline network and fleet planning problems. Based on these previous studies, we would like to use a model to solve the network flow problem of airline system for dynamic rearrangement, which might control the epidemic and bring less economic losses. 
\begin{itemize}
    \item \textbf{Network flow problem}
    Network flow is a network that satisfies the following properties. Each edge has a maximum capacity $C^{max}$, which is the maximum flow that the edge can accommodate. $C$ is the actual traffic flowing through the edge, and there is always $C$ less or equal to $C^{max}$.
    \item \textbf{Minimum cost maximum flow problem}
The minimum cost maximum flow problem is a typical problem in economics and management. Each path in a network is limited by cost and capacity. These kind of research problems are mainly want to find out how to select the path and allocate the traffic passing through the path from a to B can achieve the minimum cost requirement.
    \item \textbf{Osmosis model}
    Osmosis can be described by a phenonomenon where the spontaneous net of solvent molecules moves through a selectively permeable membrane into a region of higher concentration, in the direction that tends to equalize the two-sided solute concentration. It can also demonstrate a physical process in which any solvent moves across a selectively permeable membrane. In this way, the infection rate between two selective cities can be defined as different levels of solute concentration while flight movements will be regarded as the process of solvent molecules osmosis.   
\end{itemize}