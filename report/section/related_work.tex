\section{Proposed Model}

% \subsection{General Model}

Let us focus on the simplest situation. Considering a group of independent cities, we can draw a simple graph to describe the connectivity between cities, which means the direction and capacity of airlines. 
\begin{figure}[H]
    \centering
    \includegraphics[width=0.7\columnwidth]{pic/graph1.pdf}
    \caption{The airline graph of a city group}
    \label{fig:graph1}
\end{figure}
To further illustrate how these methods applied with empirical data, a general setting of variables and parameters is given as follows, where $i,j$ respectively represents different city or area: 
% \begin{spacing}{1}
\begin{itemize}
    \item $c_{ij}$: the capacity of each airline from city $i$ to $j$
    
    $c_{ij} \triangleq$ $f_{ij} \cdot p_{ij}$ ($f_{ij}$: average daily flights from city $i$ to $j$; $p_{ij}$: average passenger capacity per flight. Note that $p_{ij}$ is hard to attained, and in simplified cases it's set as 1)
    
    \item $d_{ij}$: the weighted value for each airline which depends on the related demand and importance
    
    $d_{ij} \triangleq \frac{t_i}{\sum_k t_k} + \frac{g_j}{\sum_k g_k} \neq d_{ji} $ ($t_i$: total annual passenger flow of the whole airport city $i$; $g_i$: annual GDP of city $i$ ) % $D$: a normalization parameter) 
    
    \item $u_{ij}$: the utility of each airline
    $u_{ij} \triangleq c_{ij} \cdot d_{ij}$
    
    \item $S_i$: the severity of the epidemic in city $i$
    
    $S_i \triangleq I_i - E_i$ ($I_i$: average number of infections in city $i$ per day; $E_i$: average number of patients cured in city $i$ per day)
    
    \item $r_{ij}$: the rate of infection (a softmax function)
    
    $r_{ij} \triangleq e^{S_i \cdot c_{ij} / T} / (e^{S_i \cdot c_{ij} / T} + e^{S_j \cdot c_{ji} / T})$ ($T$: hyper-parameter)
    
    \item $\epsilon$: the threshold of infection rate
\end{itemize}
% \end{spacing}

Specifically, for those big cities like Beijing and Shanghai, the demand and importance of this airline $d_{ij}$ must be higher than most of cities. For different airlines, if we need to reduce the same amount for capacity $c_{ij}$, the related utility must decrease more for those airlines with higher $r_{ij}$. Thus, $u_{ij} = c_{ij} \cdot d_{ij}$ makes sense intuitively. 

Besides, the definition of $r_{ij}$ can also be explained intuitively. $r_{ij}$ is regarded as the risk of epidemic transmission, which is related to $S_i,S_j$ (the severity of the epidemic situation in the city $i,j$) and $c_{ij},c_{ji}$ (the capacity of airline $i \leftrightarrow j$). Hence, $r_{ij} = f(S_i, S_j, c_{ij}, c_{ji})$, which can be directly transformed into $r_{ij} = S_i \cdot c_{ij} / (S_i \cdot c_{ij} + S_j \cdot c_{ji})$.

To polarize the effect of $S_i \cdot c_{ij}$ and ensure the sign of $r_{ij}$ is positive (considering $S_i$ could be negative if $I_i < E_i$), we apply a softmax function $r_{ij} \triangleq e^{S_i \cdot c_{ij} / T} / (e^{S_i \cdot c_{ij} / T} + e^{S_j \cdot c_{ji} / T})$, which is the definition of $r_{ij}$. Note that the $T$ here is to adjust the polarization effect. With larger T we will have smaller polarization effect.

To derive the specific form of these variables, we need to selectively choose some datasets $\{ f_{ij},p_{ij}, I_i, E_i, t_i, g_i \}$ in real world as the basis of quantitative variables $\{ c_{ij},d_{ij}, r_{ij} \}$. 

Thus, it is natural to formulate a constrained maximization optimization problem:
\begin{align*}
    \max_{c_{ij}}~& U = \sum_{i,j} u_{ij} = \sum_{i,j} c_{ij}d_{ij}\\
    \text{subject to}~~& c_{ij} \in [0, c_{ij}^{max}],~~ \forall i \neq j \\
    & r_{ij} \le \epsilon,~~ \forall i \neq j 
\end{align*}

In order to avoid nonlinear constraints, we need to simplify the condition $r_{ij} \le \epsilon$, and then we obtain the linear constraints for $\{c_{ij}\}$:
\begin{equation*}
    \begin{aligned}
     & r_{ij} \le \epsilon,~~ \forall i \neq j \\
     \Leftrightarrow~& e^{S_i \cdot c_{ij} / T} / (e^{S_i \cdot c_{ij} / T} + e^{S_j \cdot c_{ji} / T}) \le \epsilon,~~ \forall i \neq j \\
     \Leftrightarrow~& (1-\epsilon) e^{S_i \cdot c_{ij} / T} \le \epsilon e^{S_j \cdot c_{ji} / T},~~ \forall i \neq j \\
     \Leftrightarrow~& \ln{(1-\epsilon)} + \cdot S_i \cdot c_{ij} / T \le \ln{\epsilon} + S_j \cdot c_{ji} / T,~~ \forall i \neq j \\
     \Leftrightarrow~& S_i \cdot c_{ij} - S_j \cdot c_{ji} + T \cdot \ln{\frac{1-\epsilon}{\epsilon}} \le 0,~~ \forall i \neq j
    \end{aligned}
\end{equation*}

% \textbf{Model 1: Minimum Cost Problem} 

Like fusing command by civil aviation, if the city $s$ has a serious epidemic situation, then we need to cut some airlines from the figure \ref{fig:graph1} such that
after cutting, there is no path from $s$ to $t$ (eg. capital). The cost of removing airline $i\rightarrow j$ is equal to its capacity $c_{ij}$. Thus, the problem is changed into a minimum cut problem to find a cut strategy with minimum total cost. 


% \textbf{Model 2: SI model} 

% We can choose appropriate epidemic model like SI/SIR/SIRS/SEIR models to simulate the actual situation under epidemic environment. Then rate of infection $r_{ij}$ can be described by differential equations.


% \subsection{Special Models}


\subsection{Algorithm}
